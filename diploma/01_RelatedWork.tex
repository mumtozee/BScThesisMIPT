\chapter{\centering\normalsize{ОСНОВНАЯ ЧАСТЬ}}
\addcontentsline{toc}{section}{ОСНОВНАЯ ЧАСТЬ}

\section{Релевантные исследования}
К представленному в данной работе исследованию имеются несколько смежных задач и направлений. Ниже описан краткий обзор по каждому направлению и существующих подходов решения поставленных там задач.

\textbf{Извлечение персональных атрибутов}  Большинство работ по извлечению персональных характеристик из естественного языка использовали технику distant supervision и эвристические методы и шаблоны (\cite{pappu-rudnicky-2014-knowledge}, \cite{tigunova_2019}, \cite{gtky}, \cite{mazare-etal-2018-training}), у которых довольно низкая полнота (recall). В исследованиях в этой работе так же используется distant supervision. Тем не менее, он не критичен для работоспособности framework'a и служит исключительно как способ оценить качество представленных подходов и сравнить их с другими методами. В \cite{personachat} представлен датасет, состоящий из повседневных диалогов с описаниями персон каждого собеседника из 5-6 предложений, и ставится задача генерации ответа по входному контексту диалога используя персону собеседника и собственную персону. Тем не менее, выделение релевантных частей персоны к определенной реплике не является основным фокусом исследования. Аналогичная работа была проделана в статье \cite{focus}: в описанной архитектуре модели используется компонента retrieve'a нужного участка персоны, которая наиболее релевантна к текущему контексту диалога. Тем не менее, эти работы не ориентируются непосредственно извлекать эти атрибуты из диалогов, а использовать их в генерации ответа неявным образом. Наиболее близкими исследованиями к данной работе являются \cite{gtky} и \cite{genre}. В \cite{gtky} представляется архитектура модели состоящая из двух компонент: классификатор отношений и генератор сущностей, и обе обучаются в сквозном режиме как одно целое. В одном из подходов решения задачи в данной работе так же было решено придерживаться этой архитектуры, однако позже будет показано, что эти компоненты могут быть не связаны друг с другом для достижения хорошего качества. Стоит так же отметить, что в отличие от этого подхода, подход представленный в данной работе использует модели основанные на трансформер \cite{vaswani2017}, что заметно улучшает качество. Другой подход, описанный в \cite{genre} так же использует две компоненты: одну для генерации триплетов, и другую - для оценки релевантности сгенерированного триплета к реплике. Задачу извлечения признаков авторы \cite{genre} разделяют на две подзадачи: на явное извлечение атрибутов - когда некоторое отношение является подстрокой в реплике, и неявное - когда отношения нужно выводить из реплики основываясь на семантику. Хотя подход показал себя хорошо в указанных подзадачах, в общем случае, неизвестно под какую из этих подзадач подходит входная реплика.

\textbf{Построение графов знаний}   Установленный формат в котором извлекается информация в данной работе очень похож на структуру графов знаний, в которых граф так же хранится в виде кортежей из трех элементов: две сущности, и отношение между ними. В данной работе, кортежи извлекаются с помощью языковых моделей, включая авторегрессионные, которые были использованы в заполнении графов знаний (\cite{bosselut-etal-2019-comet}). В \cite{alt2019improving} использовалась модель GPT \cite{gpt2} для классификации отношения по заданным двум сущностям. Тем не менее, в данной работе авторегрессионная модель работает в обратном режиме, то есть по заданной реплике и отношению, она генерирует две сущности находящиеся в реплике в данном отношении.

\textbf{Извлечение информации}  В обработке естественного языка очень важны и хорошо изучены подходы извлечения информации в открытой и закрытой форме (\cite{banko2007}, \cite{wu-weld-2010-open}, \cite{berant-etal-2011-global}, \cite{fader2014}). Научным сообществом были представлены методы основанные, и на шаблонах (\cite{mausam-etal-2012-open}, \cite{delcorro2013}), и на обучаемых моделях (\cite{zeng-etal-2014-relation}, \cite{xu-etal-2015-classifying}, \cite{stanovsky-etal-2018-supervised}, \cite{vashishth-etal-2018-reside}), однако большинство из этих подходов извлекают информацию путем проставления тегов на части предложения. Дополнительно можно считать, что задача поставленная в этой работе относится к семейству задач отслеживания состояния диалога (dialogue state tracking \cite{dstc_survey}).
