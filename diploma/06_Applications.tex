\section{Применения}

У извлечения пользовательских атрибутов из диалогов в структурированном формате имеется потенциально очень много приложений. Например, можно использовать в поисковиках, в рекомендательных системах друзей, в социальных науках, в персонализованных диалоговых ассистентах и так далее. Причинами проведения данной работы послужили следующие применения: персонализованных диалоговые ассистенты и рекомендательные системы.

\textbf{Персонализованные диалоговые ассистенты}    Огромное внимание уделяется разработке систем, которые могут повысить качество беседы и сделать ее более увлекательной для участников. Есть два направления развития агентов персонализированного диалога: придание агентам индивидуальности как в статье про PersonaChat \cite{personachat}, или адаптация агентов к их конечным пользователям с помощью пользовательских атрибутов. Следовательно, если можно снабдить диалоговую систему модулем извлечения пользовательских атрибутов, это станет шагом в развитии персонализированных диалоговых систем.

Диалоговая система может рассматривать пользовательские атрибуты, извлеченные из истории диалога, как долгосрочную память. Эта информация позволяет избежать системе повторять одни и те же или похожие вопросы. Например, если пользователь упомянул, что он родился в сентябре 2009 года в предыдущем разговоре два дня назад, система персонализированного диалога должна избегать задавать вопросы типа "В каком месяце у вас день рождения?" или "А сколько тебе лет?". Кроме того, такие атрибуты можно использовать для фильтрации ответов системы или для предложения ответов получше. Например, для персонализированной системы было бы неуместно спрашивать "Как у вас дела в универе?", если пользователь родился в 2013м году, в то время как сейчас 2023й год. Лучше будет, если система ответит "Вау! Скоро тебе исполнится 10!" после получения информации о времени рождения пользователя.

\textbf{Персонализованные рекомендательные системы}     Существует три основных типа рекомендательных систем: knowledge-based система содержит атрибуты пользователя и продуктов и дает рекомендации на основе сходства этих атрибутов; content-based система рекомендует товары, похожие тем, которые понравились данному пользователю в прошлом, независимо от предпочтений других пользователей; в то время, как система на основе collaborative-filtering делает рекомендации основываясь на прошлых взаимодействиях всей базы пользователей, например, рассматривая k-ближайших соседей данного пользователя.

Большинство этих рекомендательных систем требуют фактического взаимодействия пользователей с продуктами, такими как щелчок мышью или просмотр. Используя решения из данной работы можно собирать пользовательские атрибуты в неявном режиме, которые затем можно использовать в кластеризации пользователей или для сохранения продуктов упомянутые пользователем в прошлом. Например, если имеются два пользователя и оба из Москвы и любят хоккей, то можно порекомендовать игру ЦСКА одному пользователю, если другой часто упоминает ее.
